%%% programmers_guide.tex --- 

%% Author: Stephen Quinney <squinney@inf.ed.ac.uk>
%% Version: $Id: programmers_guide.tex,v 0.0 2017/12/05 11:39:27 squinney Exp$


\documentclass[11pt,a4paper,titlepage]{article}
\usepackage{minted}
\usepackage{tabulary}
\usepackage{datetime}

\setlength{\voffset}{0in}
\setlength{\topmargin}{0in}
\setlength{\headsep}{0in}
\setlength{\textwidth}{6.26in}
\setlength{\evensidemargin}{0in}
\setlength{\oddsidemargin}{0in}
\setlength{\headheight}{0in}
\setlength{\textheight}{9.5in}

\title{LCFG Core : C Programmers Guide}
\author{Stephen Quinney \texttt{<squinney@inf.ed.ac.uk>}}
\date{\today}

\begin{document}

\maketitle

\tableofcontents
\clearpage

\section{Introduction}

This is a guide for programmers who wish to use the LCFG Core C
libraries. The aim is not to provide exhaustive detail on all available
functions but rather the intention is to provide an overview of the
most commonly useful facilities. 

Complete API docs are available in html and PDF formats in
\texttt{/usr/share/doc/lcfg-core-doc}. On most Unix systems the header
files are stored in either \texttt{/usr/include/lcfg} or
\texttt{/usr/local/include/lcfg}

\section{Resources}

At its most simple an LCFG resource can be considered to be a mapping
between a name and a value but there are a number of other aspects
which represent value type, context and derivation information. A
resource is represented using an \texttt{LCFGResource} structure which
has the following attributes:

\begin{center}
  \begin{tabular}{| l l p{7cm}|}
\hline
Attribute  & Variable Type & Notes \\ 
\hline
name       & string & Name (e.g. ``foo'') - \textit{required}\\
value      & string & Value \\
template   & \texttt{LCFGTemplate} & A set of templates used for creating sub-resource names for tag lists \\
context    & string & Context expression for when resource is valid \\
derivation & string & Information on where in sources resource value was modified \\
comment    & string & Type comments \\
type       & \texttt{LCFGResourceType} & Type ID (i.e. for string, integer, boolean, list) \\
priority   & integer & Priority is derived from evaluation of context expression \\
\hline
\end{tabular}
\end{center}

Creating a new resource structure from scratch is done using the
\texttt{lcfgresource\_new} function. This is straightforward and can
be done like this:

\begin{minted}[frame=single]{c}
LCFGResource * res = lcfgresource_new();

char * resname = strdup("foo");
lcfgresource_set_name( res, resname );

char * resvalue = strdup("bar");
lcfgresource_set_value( res, resvalue );

/* Avoid memory leaks */

lcfgresource_relinquish(res);
\end{minted}

In this example it is particularly important to note that the
\textit{name} and \textit{value} strings are duplicated before being
passed to the set functions. When a string is passed to any of these
functions the resource structure assumes ``ownership'' of the
memory. If a resource structure is no longer required it should be
destroyed to avoid memory leaks (using
\texttt{lcfgresource\_relinquish}) and this will also free the
memory for the string attributes. One consequence of this is that only
strings which are allocated on the heap can be used, values on the
stack will cause weird behaviour.

\subsection{Attributes}

\subsubsection{Name}
\label{subsec:res_name}

To be considered valid a resource MUST have a name. The first
character of a resource name MUST be in the set \mintinline{c}|[A-Za-z]| and
all subsequent characters MUST be in the set
\mintinline{c}|[A-Za-z0-9_]|. The main reason for this restriction is so that
the resource name can be safely used as a variable name in shell
scripts and in templates. If a name is invalid the
\texttt{lcfgresource\_set\_name} function will return a false value
and the \texttt{errno} will be set to \texttt{EINVAL}.

The following functions are available for accessing and manipulating
the resource name:

\begin{minted}[frame=single]{c}
bool lcfgresource_valid_name( const char * name );
bool lcfgresource_has_name( const LCFGResource * res );
const char * lcfgresource_get_name( const LCFGResource * res );
bool lcfgresource_set_name( LCFGResource * res, char * new_value );
\end{minted}

It is important to note that the string returned by
\texttt{lcfgresource\_get\_name} is not a copy. It should NOT be
modified since that will also affect the name for the resource, to
change the name of a resource use the \texttt{lcfgresource\_set\_name}
function. Note that when a string is passed in as the name the
resource will assume ``ownership'' and the memory will be freed when
the resource is destroyed.

\subsubsection{Value}
\label{subsec:res_value}

A resource value is validated according to the current type for the
resource (see \ref{subsec:res_type} for full details). By default a resource
value is considered to be a string which can hold any value. If a value
is invalid the \texttt{lcfgresource\_set\_value} function will return a
false value and the \texttt{errno} will be set to \texttt{EINVAL}.

The following functions are available for accessing and manipulating
the resource value:

\begin{minted}[frame=single]{c}
bool lcfgresource_valid_value( const LCFGResource * res,
                               const char * value );
bool lcfgresource_has_value( const LCFGResource * res );
const char * lcfgresource_get_value( const LCFGResource * res );
bool lcfgresource_set_value( LCFGResource * res, char * new_value );
bool lcfgresource_unset_value( LCFGResource * res );
\end{minted}

The type checking does not permit \texttt{NULL} as a valid value so
the \texttt{lcfgresource\_unset\_value} function is particularly useful
when it is necessary to reset a resource value (which frees any memory
for the previous value string).

It is important to note that the string returned by
\texttt{lcfgresource\_get\_value} is not a copy. It should NOT be
modified since that will also affect the value for the resource, to
change the value of a resource use the \texttt{lcfgresource\_set\_value}
function. Note that when a string is passed in as the value the
resource will assume ``ownership'' and the memory will be freed when
the resource is destroyed.

\subsubsection{Type}
\label{subsec:res_type}

There are currently only a few resource \textit{types} supported by
LCFG, these are \textit{boolean}, \textit{integer}, \textit{list} and
\textit{string} (the default). These \textit{types} are used to
restrict the set of valid values for a resource. This is not a
full-blown type system and internally the resource values are always
stored as strings with the type checking actually being just simple
checks on the characters in the string.

\begin{enumerate}

\item \textbf{String} This is the default type which may hold any
  value. An LCFG component schema supports further limiting the
  permitted values for a string, that is currently only handled by
  the server as the information is not exported to the client.

\item \textbf{Boolean} This can be used to represent a binary state
  (e.g. true/false, yes/no, on/off). The valid values are \mintinline{c}|""|
  (empty string) for false values and \mintinline{c}|"yes"| for true values. A
  \texttt{lcfgresource\_canon\_boolean} function is provided which can
  be used to translate various other supported strings to these two
  standard strings.

\item \textbf{Integer} This can be used to represent an integer value,
  note that internally it is still stored as a string.

\item \textbf{List} This can be used to hold a space-separated list of
  LCFG \textit{tags}. Note that this is NOT a generic list structure,
  that can usually be simulated using a string with space-separated
  values. All characters in a \textit{tag} string MUST be in the set
  \mintinline{c}|[A-Za-z0-9_]|. Typically for each tag there will be associated
  sub-resources, the names of which are assembled using the templates
  for the parent resource.

\end{enumerate}

The current type of a resource can be easily checked using one of the
following functions. There is also support for checking if a value is
true (which may vary depending upon the type).

\begin{minted}[frame=single]{c}
bool lcfgresource_is_string(  const LCFGResource * res );
bool lcfgresource_is_integer( const LCFGResource * res );
bool lcfgresource_is_boolean( const LCFGResource * res );
bool lcfgresource_is_list(    const LCFGResource * res );
bool lcfgresource_is_true( const LCFGResource * res );
\end{minted}

To check if a string holds a valid value for the type the following
functions are provided:

\begin{minted}[frame=single]{c}
bool lcfgresource_valid_boolean( const char * value );
bool lcfgresource_valid_integer( const char * value );
bool lcfgresource_valid_list(    const char * value );
bool lcfgresource_valid_value( const LCFGResource * res,
                               const char * value );
\end{minted}

Note that there is no function for checking the validity of
strings. The \texttt{lcfgresource\_valid\_value} function supports
checking if a value is acceptable for a particular resource before an
attempt is made to actually set the value. That is often a good way to
make invalid value handling somewhat simpler.

The following functions are available for accessing and manipulating
the resource type:

\begin{minted}[frame=single]{c}
LCFGResourceType lcfgresource_get_type( const LCFGResource * res );
bool lcfgresource_set_type( LCFGResource * res,
                            LCFGResourceType new_type );
char * lcfgresource_get_type_as_string( const LCFGResource * res,
                                        LCFGOption options );
bool lcfgresource_set_type_as_string( LCFGResource * res,
                                      const char * new_type,
                                      char ** msg );
\end{minted}

Note that if a resource already has a value then the type can only be
changed if the value is valid for the new type. If a type
is invalid the \texttt{lcfgresource\_set\_type} function will return a
false value and the \texttt{errno} will be set to \texttt{EINVAL}. 

The \texttt{LCFGResourceType} is an enum which holds the following
values:

\begin{minted}[frame=single]{c}
LCFG_RESOURCE_TYPE_STRING    /* String, can hold any value */
LCFG_RESOURCE_TYPE_INTEGER   /* Integer */
LCFG_RESOURCE_TYPE_BOOLEAN   /* Boolean */
LCFG_RESOURCE_TYPE_LIST      /* List of tag names */
LCFG_RESOURCE_TYPE_PUBLISH   /* Published to spanning map, like String */
LCFG_RESOURCE_TYPE_SUBSCRIBE /* Subscribed from spanning map, like String*/
\end{minted}

The \texttt{lcfgresource\_get\_type\_as\_string} function can be used
to assemble a new type string which will contain not just the type
name but also any associated comments and, for lists, any
templates. The inclusion of templates can be avoided by specifying the
\texttt{LCFG\_OPT\_NOTEMPLATES} option. Note that when no longer
required the memory for the generated type string will need to be
freed.

Typically the type for a resource will be set using the
\texttt{lcfgresource\_set\_type\_as\_string} function and that will
also process any templates (see \ref{subsec:res_template}) and comments (see
\ref{subsec:res_comment}). A type string may begin with the $\%$ (percent)
symbol. After the type name there may be a comment string enclosed in
parentheses. For list types there will typically also be a list of
templates. The following strings are valid type declarations:

\begin{verbatim}
%boolean
string(comment string describing value)
list: run_$ user_$ args_$ object_$ method_$
\end{verbatim}

\subsubsection{Template}
\label{subsec:res_template}

If a resource is an LCFG tag list then it may have a list of
associated templates which are used to generate the names of
sub-resources. For example the \texttt{ng\_plugins} tag list resource
(which is part of \textit{ngeneric}) is defined like this:

\begin{verbatim}
@ng_plugins ng_plugin_params_$ ng_plugin_methods_$ ng_plugin_module_$
ng_plugins
\end{verbatim}

It has templates for 3 sub-resource names, the \$ (dollar) placeholder
character is replaced with the tag name. For example a tag named
\texttt{foo} would result in sub-resources named:

\begin{verbatim}
ng_plugin_params_foo
ng_plugin_methods_foo
ng_plugin_module_foo
\end{verbatim}

The complete set of functions which can be used for manipulating LCFG
templates can be found in the \texttt{lcfg/templates.h} header
file. The basic functions are:

\begin{minted}[frame=single]{c}
bool lcfgresource_has_template( const LCFGResource * res );
LCFGTemplate * lcfgresource_get_template( const LCFGResource * res );
bool lcfgresource_set_template( LCFGResource * res,
                                LCFGTemplate * new_value );
char * lcfgresource_get_template_as_string( const LCFGResource * res );
bool lcfgresource_set_template_as_string( LCFGResource * res,
                                         const char * new_value,
                                         char ** msg );
\end{minted}

A sub-resource name can be assembled using the
\texttt{lcfgresource\_build\_name} function. Here is an example which
loads an \texttt{LCFGTemplate} structure from a string and then uses
it to build a child resource name.

\begin{minted}[frame=single]{c}
const char * tmpl_str =
  "ng_plugin_params_$ ng_plugin_methods_$ ng_plugin_module_$";

LCFGStatus rc = LCFG_STATUS_OK;
LCFGTemplate * tmpl = NULL;
char * msg = NULL;

rc = lcfgtemplate_from_string( tmpl_str, &tmpl, &msg );

const char * tags_str = "foo";
LCFGTagList * taglist = NULL;
rc = lcfgtaglist_from_string( tags_str, &taglist, &msg );

const char * field = "ng_plugin_params"; /* base of name */
char * resname = lcfgresource_build_name( tmpl, taglist, field,
                                          &msg );
printf( "Resource name is '%s'\n", resname );

/* Avoid memory leaks */

lcfgtaglist_relinquish(taglist);
lcfgtemplate_destroy(tmpl);
free(resname);
free(msg);
\end{minted}

This will result in a sub-resource name of
\texttt{ng\_plugin\_params\_foo}. There can be multiple levels of
placeholders (e.g. a template like \mintinline{c}|foobar_$_$_$|) in
which case the tag list needs to contain sufficient tags for each
placeholder. It is also important to note that the tag list is
processed backwards so a tag list of \mintinline{c}|a,b,c| would give
a sub-resource name of \mintinline{c}|foobar_c_b_a|.

\subsubsection{Comment}
\label{subsec:res_comment}

\begin{minted}[frame=single]{c}
bool lcfgresource_has_comment( const LCFGResource * res );
const char * lcfgresource_get_comment( const LCFGResource * res );
bool lcfgresource_set_comment( LCFGResource * res, char * new_value );
\end{minted}

\subsubsection{Context}
\label{subsec:res_context}

\begin{minted}[frame=single]{c}
bool lcfgresource_valid_context( const char * expr );
bool lcfgresource_has_context( const LCFGResource * res );
const char * lcfgresource_get_context( const LCFGResource * res );
bool lcfgresource_set_context( LCFGResource * res, char * new_value );
bool lcfgresource_add_context( LCFGResource * res,
                               const char * extra_context );
\end{minted}

\subsubsection{Priority}
\label{subsec:res_priority}

\begin{minted}[frame=single]{c}
int lcfgresource_get_priority( const LCFGResource * res );
char * lcfgresource_get_priority_as_string( const LCFGResource * res );
bool lcfgresource_set_priority( LCFGResource * res, int priority );
bool lcfgresource_set_priority_default( LCFGResource * res );

bool lcfgresource_is_active( const LCFGResource * res );

bool lcfgresource_eval_priority( LCFGResource * res,
                                 const LCFGContextList * ctxlist,
				 char ** msg );
\end{minted}

\subsubsection{Derivation}
\label{subsec:res_derivation}

\begin{minted}[frame=single]{c}
bool lcfgresource_has_derivation( const LCFGResource * res );
const char * lcfgresource_get_derivation( const LCFGResource * res );
bool lcfgresource_set_derivation( LCFGResource * res, char * new_value );
bool lcfgresource_add_derivation( LCFGResource * resource,
                                  const char * extra_deriv );
\end{minted}

\subsection{Import}

Resource information can be loaded from various formats. Functions are
provided for parsing resource specification strings
(\ref{subsec:res_spec}) and loading resource information from the
environment (\ref{subsec:res_env}).

\subsubsection{Specification string}
\label{subsec:res_spec}

Typically new \texttt{LCFGResource} structures are created from a
specification string which contains information on the resource name
and value in the standard form \texttt{key=value}. The key may also
contain information on the node and component names in the form
\texttt{node.comp.res}. The following are all valid resource
specifications:

\begin{verbatim}
host.client.ack=yes
client.ack=yes
ack=yes
\end{verbatim}

These strings can all be parsed using the
\texttt{lcfgresource\_from\_spec} function. Here is an example which
reads the specification and then prints the resource information to
stdout in a different ``summary'' style.

\begin{minted}[frame=single]{c}
const char * spec = "host.client.ack=yes";

LCFGResource * res = NULL;
char * nodename = NULL;    /* 'host'   */
char * compname = NULL;    /* 'client' */
char * msg = NULL;         /* error message */

LCFGStatus rc = lcfgresource_from_spec( spec, &res,
                                       &nodename,
                                       &compname,
                                       &msg );

if ( rc == LCFG_STATUS_ERROR ) {
  fprintf( stderr, "Failed to parse spec '%s': %s\n",
                   spec, msg );
} else {
  lcfgresource_print( res, compname,
                      LCFG_RESOURCE_STYLE_SUMMARY,
                      LCFG_OPT_NONE, stdout );
}

/* Avoid memory leaks */

lcfgresource_relinquish(res);
free(nodename);
free(compname);
free(msg);
\end{minted}

If the specification does not contain information for the node or
component name the variable value will be \texttt{NULL}.

\subsubsection{Environment}
\label{subsec:res_env}

Resource information can also be loaded directly from shell
environment variables using the \texttt{lcfgresource\_from\_env}
function. This is used by LCFG components written in shell when
calling tools such as qxprof and sxprof.

For example, to load the \texttt{client.ack} resource from the environment:

\begin{minted}[frame=single]{c}
const char * compname = "client";
const char * resname = "ack";
LCFGResource * res = NULL;
LCFGStatus rc = lcfgresource_from_env( resname, compname,
                                       NULL, NULL,
                                       &res, LCFG_OPT_NONE, &msg );

if ( rc == LCFG_STATUS_ERROR ) {
  fprintf( stderr, "Failed to load %s.%s resource from environment: %s\n",
                   compname, resname, msg );
} else {
  lcfgresource_print( res, compname,
                      LCFG_RESOURCE_STYLE_SUMMARY,
                      LCFG_OPT_NONE, stdout );
}

/* Avoid memory leaks */

lcfgresource_relinquish(res);
free(msg);
\end{minted}

This checks the environment for variables which hold resource value
and type information for the given name and creates a new
\texttt{LCFGResource}. The variable names are a simple concatenation
of the resource name and any prefix specified.  By default the value
prefix will be \texttt{LCFG\_\%s\_} and the type prefix will be
\texttt{LCFGTYPE\_\%s\_}. The \texttt{\%s} is replaced with the name
of the component so in this example the function checks for
environment variables named \texttt{LCFG\_client\_ack} and
\texttt{LCFGTYPE\_client\_ack}.

If there is no environment variable for the resource value in the
environment an error will be returned unless the
\texttt{LCFG\_OPT\_ALLOW\_NOEXIST} option is specified. When this
option is specified and there is no environment variable the value is
assumed to be an empty string. Note that this may not be a valid value
for some resource types.

\subsection{Export}

Resource information can be serialised into various formats. Some of
the serialisation formats are also supported for importing data
(e.g. spec strings, status and environment) and others are for human
consumption (e.g. summary).

Most of the serialisation functions conform to a standard interface
and write to a string buffer which may be already initialised. The
buffer will be grown when necessary with the size parameter being
updated. The length of the generated string is returned or if an error
occurs a negative value will be returned. The advantage of this
approach is that a large string buffer can be pre-allocated which can
be repeatedly used for serialising a large number of resources
(e.g. when serialising data for an entire component or profile). When
no longer required the string buffer must be freed to avoid leaking
memory.

There is an \texttt{lcfgresource\_to\_string} function which provides
a standard wrapper interface to most of the serialisation
functions. The output style can be selected using the
\texttt{LCFGResourceStyle} enum.

\begin{minted}[frame=single]{c}
const char * compname = "client";

size_t size = 0;
char * result = NULL;

ssize_t length = lcfgresource_to_string( res, compname,
                                        LCFG_RESOURCE_STYLE_SPEC,
                                        LCFG_OPT_NEWLINE,
                                        &result, &size );
if ( length > 0 )
  fputs( result, stdout );
else
  fprintf( stderr, "Failed to serialise resource\n" );

free(result);
\end{minted}

The supported styles are:

\begin{minted}[frame=single]{c}
LCFG_RESOURCE_STYLE_SPEC    /**< Standard LCFG resource specification */
LCFG_RESOURCE_STYLE_STATUS  /**< LCFG status block (as used by components) */
LCFG_RESOURCE_STYLE_SUMMARY /**< qxprof style summary */
LCFG_RESOURCE_STYLE_VALUE   /**< Value only (possibly encoded) */
\end{minted}

The different serialisation functions support different options, that
value will be passed directly to the required function.

\subsubsection{Specification string}

This is the standard serialisation format for LCFG
resources. Depending on the available information it will be
serialised as a single line like \texttt{comp.res[context]=value}. The
following options are supported and can be combined:

\begin{minted}[frame=single]{c}
LCFG_OPT_NOPREFIX  /* do not include component name */
LCFG_OPT_NOVALUE   /* do not include value */
LCFG_OPT_NOCONTEXT /* do not include context */
LCFG_OPT_ENCODE    /* encode significant whitespace in value */
LCFG_OPT_NEWLINE   /* append newline character */
\end{minted}

If the component name is \texttt{NULL} it will not be ignored and the
'.'  (period) separator character will not be added. If the value is
\texttt{NULL} the '=' (equals) character will still be included. If
there is no context information the '[]' (square brackets) will not be
included. When enabled, the value encoding is done using the
\texttt{lcfgresource\_enc\_value} function which encodes newlines and
carriage returns to avoid the result appearing as a multi-line
string. For example:

\begin{minted}[frame=single]{c}
const char * compname = "client";

size_t size = 0;
char * result = NULL;

ssize_t length = lcfgresource_to_spec( res, compname,
                                      LCFG_OPT_ENCODE|LCFG_OPT_NEWLINE,
                                      &result, &size );
if ( length > 0 )
  fputs( result, stdout );
else
  fprintf( stderr, "Failed to serialise resource\n" );

free(result);
\end{minted}

\subsubsection{Status}

\subsubsection{Summary}

\subsubsection{Value}

\subsubsection{Export}

\subsubsection{Environment}

\section{Components}

\section{Packages}

\subsection{Attributes}

\section{Package Lists}

\section{Contexts}

\section{Profiles}

\end{document}
